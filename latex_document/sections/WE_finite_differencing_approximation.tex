\section{Numerical Differentiation}
\subsection{Forward, Backward and Central Difference Quotients}
In order to numerically solve a wave equation one typically approximates all the derivatives by finite differences with regard to discrete grid points.
Hence, the first step is to choose a discretization. For brevity we consider a uniform partition in the spatial dimension, so that
\[
x_i = i h  \quad \text{with } i = 0,...,n-1
\]
and stepsize $h = 1/(n-1)$ specifies the position on the domain $\Omega = [0,1] $.

Now we want to find a formula for the approximation of the derivative of the field function $u(x)$ (we neglect the temporal dependency for a moment and will tackle that later).
We consider the Taylor expansion of the field at "the next position", i.e. $u(x + h)$:
\begin{equation}
   u(x+h) = \sum_{j = 0}^{\infty} \frac{h^j}{j!} \left(\frac{\mathrm{d}^n u}{\mathrm{d} x^n} \right)_{x+h}
   = u(x) + h \frac{\mathrm{d} u}{\mathrm{d} x} + \frac{h^2}{2!} \frac{\mathrm{d}^2 u}{\mathrm{d} x^2} + \frac{h^3}{3!} \frac{\mathrm{d}^3 u}{\mathrm{d} x^3} + ...
   \label{eq:taylorfw}
 \end{equation}
Solving this for the first derivative yields
 \begin{equation}
   \frac{\mathrm{d} u}{\mathrm{d} x} = \frac{u(x+h) - u(x)}{h} - \frac{h}{2!} \frac{\mathrm{d}^2 u}{\mathrm{d} x^2} - \frac{h^2}{3!} \frac{\mathrm{d}^3 u}{\mathrm{d} x^3} - ...
 \end{equation}
Taking into account only the first term of the right-hand side, this is called the \emph{forward difference quotient}:
\begin{equation}
  \mathrm{D}^+_h u(x) = \frac{u(x+h) - u(x)}{h}
  \label{eq:discfw}
\end{equation}
Similarly, one can approximate the derivative by expanding the value $u(x-h)$ in a Taylor series. This results in the \emph{backward difference quotient}
\begin{equation}
  \mathrm{D}^-_h u(x) = \frac{u(x) - u(x-h)}{h}~.
\end{equation}
In order two minimize errors we can combine the forward and backward difference to a \emph{central difference quotient}
\begin{equation}
  \mathrm{D}_h u(x)= \frac{1}{2}[\mathrm{D}^+_h+ \mathrm{D}^-_h ] u(x)= \frac{u(x+h) - u(x-h)}{2h}~.
\end{equation}


\subsection{Error Estimations}
The discretization error is the deviation of the numerical approximation compared to the analytical value
\begin{equation}
  \mathrm{E}_h = |\frac{\mathrm{d} u}{\mathrm{d} x} - \mathrm{D}_h u(x)|~.
\end{equation}
It stems from the finiteness of step size $h$, as well as from rounding errors arising from the computer accuracy when conducting numerical calculations. In the following we want to find expressions for the former error to compare them for the various difference quotients.
In general both, forward and backward difference quotient, are consistent, i.e. for $h \rightarrow 0$ the approximated derivative approaches the analytical derivative. From the comparison of \ref{eq:taylorfw} with \ref{eq:discfw} we find that the discretization error for the forward differentiation method is
\begin{align}
  \mathrm{E}^+_h &= \frac{h}{2!} \frac{\mathrm{d}^2 u}{\mathrm{d} x^2} + \frac{h^2}{3!} \frac{\mathrm{d}^3 u}{\mathrm{d} x^3} + ... \\
  &= \mathcal{O}(h)~.
\end{align}
The same holds for the backward differentiation method.
However, for the central difference quotient we get
\begin{align}
\mathrm{E}_h &= \frac{h^2}{3!} \frac{\mathrm{d}^3 u}{\mathrm{d} x^3} + \frac{h^3}{4!} \frac{\mathrm{d}^4 u}{\mathrm{d} x^4}... \\
&= \mathcal{O}(h^2)~.
\end{align}
\subsection{Approximating the Second Derivative}
To approximate the second derivative of function $u(x)$, we recall the Taylor expansion of $u(x+h)$ and $u(x-h)$:
\begin{align}
  u(x+h) &= u(x) + h \frac{\mathrm{d} u}{\mathrm{d} x} + \frac{h^2}{2!} \frac{\mathrm{d}^2 u}{\mathrm{d} x^2} + \frac{h^3}{3!} \frac{\mathrm{d}^3 u}{\mathrm{d} x^3} + ... \\
  u(x-h) &= u(x) - h \frac{\mathrm{d} u}{\mathrm{d} x} + \frac{h^2}{2!} \frac{\mathrm{d}^2 u}{\mathrm{d} x^2} - \frac{h^3}{3!} \frac{\mathrm{d}^3 u}{\mathrm{d} x^3} + ...
\end{align}
Adding up both sides respectively and solving for the second derivative yields
\begin{equation}
  \frac{\mathrm{d}^2 u}{\mathrm{d} x^2} = \frac{u(x-h) - 2 u(x) + u(x+h)}{h^2} - \frac{h^2}{3!} \frac{\mathrm{d}^3 u}{\mathrm{d} x^3} - ...
\end{equation}

which results in the differential operator with the error of $\mathcal{O}(h^2)$:
\begin{equation}
  \mathrm{D}^2_h u(x) = \frac{u(x-h) - 2 u(x) + u(x+h)}{h^2}
\end{equation}

\subsection{Approximating to a higher order}
As we plan to use a Runge-Kutta time-integration with a error term that accumulates to $\mathcal{O}(h^4)$. Therefor it is useful to provide the derivatives with the same fourth order error term. We conduct this in the similar manner as the first and second order differential operators while simply including more terms of higher order. This leads to the following central derivative stencils:

\begin{equation}
  \mathrm{D}^1_h u(x) = \frac{-u(x+2h)+8u(x + h) - 8u(x - h)+ u(x-2h)}{12h}
\end{equation}
\begin{equation}
  \mathrm{D}^2_h u(x) = \frac{-u(x+2h) + 16u(x+h) - 30u(x) + 16u(x-h) - u(x-2h)}{12h^2}
\end{equation}

\subsection{Convergence}
In the following we show the results of the finite difference approximation in 1st and 2nd order for a set of example functions. These are:

\fontsize{10}{12}\selectfont
\begin{alignat*}{3}
	f_1 &=(x-\tfrac{1}{2})^2+x
		& f_1' &=2x
		& f_1'' &=2 \\
	f_2 &=(x-\tfrac{1}{2})^3+(x-\tfrac{1}{2})^2+x \hspace{0.3cm}
		& f_2' &= 3(x-\tfrac{1}{2})^2+2(x-\tfrac{1}{2})+1
		& f_2'' &= 6x-1\\
	f_3 &= \sqrt{x}
		& f_3' &=\tfrac{1}{2\sqrt{x}}
		& f_3'' &=-\tfrac{1}{4}x^{-\tfrac{3}{2}} \\
	f_4 &= \sin(12\pi x)
		& f_4' &= 12 \pi \cos(12\pi x)
		& f_4'' &= -(12\pi)^2 \sin(12\pi x)\\
	f_5 &= \sin^4(12\pi x)
		& f_5' &= 4 \cdot 12\pi \sin^3(12\pi x) \cos(12\pi x) \hspace{0.3cm}
		&  f_5'' &= 4  (12\pi \sin(12\pi x))^2 \\ & & & & &~~~\cdot\big(3\cos^2(12\pi x) - \sin^2(12\pi x) \big) \\
	f_6 &=g_a = \exp(-ax^2)
		& g_a' &= -2ax\exp(-ax^2)
		& g_a'' &= \exp(-ax^2) ((2ax)^2-2a)\\
\end{alignat*}
% f_5 &= \sin^4(12\pi x)
%   & f_5' &= 4 \cdot 12\pi \sin^3(12\pi x) \cos(12\pi x)
%   &  \begin{split}f_5''=
%   &4  (12\pi)^2
%   \\&\big(3\sin^2(12\pi x)\cos^2(12\pi x) - \sin^4(12\pi x) \big) \end{split}\\
% f_6=g_a &= \exp(-ax^2)
%   & g_a' &= -2ax\exp(-ax^2)
%   & g_a'' &= \exp(-ax^2) ((2ax)^2-2a)\\
\fontsize{12}{14}\selectfont

We call a differential Operator $D$ consistent of order k if,
\begin{equation}
	something
\end{equation}
If a sequence $x_1,x_2,...,x_n$ converges to a value $r$ and if there exist real numbers $\lambda >0 $ and $\alpha \geq 1$ such that
\begin{equation}
	\lim_{n \to \infty} \frac{|x_{n+1}-r|}{|x_n-r|^\alpha}= \lambda
\end{equation} 
then we say that $\alpha$ is the \textbf{order} and $\lambda$ the \textbf{rate} of convergence of the sequence.


