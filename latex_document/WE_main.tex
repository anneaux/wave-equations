\documentclass[a4paper,headsepline,12pt,oneside]{scrartcl}

\usepackage{scrlayer-scrpage}
\usepackage{graphicx}
\usepackage{pdfpages}
\graphicspath{{./graphics/}{./graphics/convergence/convergence_to_analytical_derivative/}}
\usepackage[format=plain,bf, indention=1.5cm]{caption}
\usepackage{subcaption}
\usepackage[subfigure,titles]{tocloft}
\usepackage{placeins}		%allows FloatBarrier command
\usepackage{import}


\usepackage[utf8]{inputenc}
\usepackage[T1]{fontenc}
\usepackage{lmodern}
\usepackage[english]{babel}


\usepackage{float}
\usepackage{multirow}
\usepackage{epstopdf}		%converts every .eps-file to a .pdf-file, which can then be included
\usepackage{booktabs}		%table_package
\usepackage{url}


\usepackage{geometry}
\geometry{lmargin=2.5cm, rmargin=2.0cm ,bmargin = 2cm}
\setlength{\parindent}{0em}
\setlength{\parskip}{0.5em}


\usepackage{physics}
\usepackage{amsmath, amssymb, amsthm}
\usepackage[version=4]{mhchem}
\newcommand{\s}[1]{\mathrm{#1}}


\automark*{section}
\clearpairofpagestyles
\ihead{\headmark}
\ohead{\pagemark}


\usepackage[numbers,square,sort&compress]{natbib}
\usepackage{sectsty}
\fontfamily{lmr}\selectfont
\allsectionsfont{\fontfamily{lmr}\selectfont}
%colorful linking within a PDF-Viewer
\usepackage{hyperref}
\hypersetup{
    colorlinks=true,
    linkcolor=blue,
    filecolor=cyan,
    urlcolor=cyan,
    citecolor=green
}

% Packages and settings for code input
\usepackage{listings}
% Python style for highlighting
\definecolor{light-gray}{gray}{0.92}
\definecolor{deepgreen}{rgb}{0,0.5,0}
\definecolor{deepblue}{rgb}{0,0,0.8}
\lstset{frame=tb,
  language=Python,
  aboveskip=3mm,
  belowskip=3mm,
  showstringspaces=false,
  columns=flexible,
  basicstyle={\small\ttfamily},
  numbers=none, %line numbers
  numberstyle=\tiny\color{gray},
  keywordstyle=\color{deepblue},
  otherkeywords={self},
  commentstyle=\color{deepgreen},
  stringstyle=\color{deepblue},
  breaklines=true,
  breakatwhitespace=true,
  tabsize=3,
  backgroundcolor = \color{light-gray}
}

% % % Python environment
% \lstnewenvironment{python}[1][]{
% \pythonstyle
% \lstset{#1}
% }{}
% % Python for external files
% \newcommand\pythonexternal[2][]{{
% \pythonstyle
% \lstinputlisting[#1]{#2}}}
% % Python for inline
% \newcommand\pythoninline[1]{{\pythonstyle\lstinline!#1!}}


\usepackage{lipsum}
%%%%%%%%%%%%%%%%%%%%%%%%%%%%%%%%%%%%%%%%%%%%%%%%%%%%%%%%%%%%%%
%%%%%%%%%%%%%%%%%%%%%%%%%%%%%%%%%%%%%%%%%%%%%%%%%%%%%%%%%%%%%%
\begin{document}
%\maketitle
%\thispagestyle{empty}
%\clearpage
%%%%%%%%%%%%%%%%%%%%%%%%%%%%%Titelseite%%%%%%%%%%%%%%%%%%%%%%%%%%%%%
% \begin{titlepage}
% 	\centering
%   \hrule \vspace{1cm}
% 	{\Large \textbf{Solving the Initial Boundary Value Problem for the Wave Equation} \par}
% 	\vspace{1cm} \hrule
% 	{\scshape\Large Report on Research Labworks 2020}
% 	\vspace{2cm}
%   \begin{figure}[H]
%   \center
%   \includegraphics[width=0.5\textwidth]{graphics/fsulogo.jpg}
%   \end{figure}
% 	{}
% 	\vspace{2cm}
% 	% {\scshape\large Friedrich-Schiller-Universität Jena\par}
% 	{\scshape\large Physikalisch-Astronomische Fakultät\par}
% 	\vfill
% 	{\Large Anne Weber and Johannes Nicklaus\par}
% 	\vfill
% 	{Student's numbers: 162587 and 170099 \\
%   Supervisor: Vsevolod Nedora\par}

% 	\vfill

% % Bottom of the page
% 	{\large \today\par}
% \end{titlepage}



\begin{titlepage}

\thispagestyle{empty}
    \makeatletter
    \centering
    %\vspace*{0.5 cm}
    \includegraphics[scale = 0.2]{graphics/fsulogo.jpg}\\[1.0 cm]
    \textsc{\Large Physikalisch-Astronomische Fakultät\\Theoretisch-physikalisches Institut}
    \vspace{1cm} \hrule \vspace{0.4cm}
    \Large \textbf{Solving the Initial Boundary Value Problem for the Wave Equation}
    \vspace{0.4cm} \hrule \vspace{1cm}
    \Large Report on Research Labworks \\
    submitted on \today \\
    \vspace{1cm}
    {\centering
    %\includegraphics[width = 0.45 \textwidth]{graphics/coolesbild.png}
    }

    \begin{minipage}[t]{0.4\textwidth}
        \begin{flushright} \large
            \emph{Supervisor:}\\
            \emph{Students:}
         \end{flushright}
    \end{minipage}%hier war vorher noch eine tilde die hab ich mal weggemacht
    \hspace{0.5cm}
    \begin{minipage}[t]{0.4\textwidth}
    \begin{flushleft} \large
          Vsevolod Nedora \\
          Johannes Nicklaus (170099) \\ Anne Weber (162587)
    \end{flushleft}
    \end{minipage}\\[2 cm]
    % \hfill
    \makeatother
\end{titlepage}







%%%%%%%%%%%%%%%%%%%%%%%%%%%%Gutachterseite%%%%%%%%%%%%%%%%%%%%%%%%%%


%%%%%%%%%%%%%%%%%%%%%%%%%%%%Inhaltsangabe%%%%%%%%%%%%%%%%%%%%%%%%%%%
\newpage
\setcounter{page}{0}
\tableofcontents
%%%%%%%%%%%%%%%%%%%%%%%%%%%%%%%%%%%%%%%%%%%%%%%%%%%%%%%%%%%%%%%%%%%%
\clearpage
\setcounter{page}{1}

\import{sections/}{WE_general_analytical_solution}

\clearpage

\import{sections/}{WE_finite_differencing_approximation}
\clearpage

\import{sections/}{WE_Runge_Kutta_time_integration}

% \import{sections/}{WE_solvingWE}

\clearpage
%\import{sections/}{WE_periodic_boundary_condition}
\clearpage
\appendix
\import{sections/}{WE_app_derivatives_of_example_functions}
\import{sections/}{WE_rungekutta}



\end{document}