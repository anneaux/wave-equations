\section{Theoretical Background}
\subsection{General Analytical Solution}

The wave equation (WE) for the scalar field $\phi (t,x) $ in one spatial and one time dimension is
\begin{equation}
	0 =  \frac{\partial^2 \phi}{\partial t^2}- c^2 \frac{\partial^2 \phi}{\partial x^2} = \partial_{tt} \phi -c^2 \partial_{xx} \phi
	\label{eq:we}
\end{equation}
with $c \in \mathbb{R}$.
According to the symmetry of second derivatives this can be divided into two factors
\begin{equation}
	0 = \left(\partial_{t} - c \partial_{x} \right)
	\left( \partial_{t} + c \partial_{x} \right)  \phi ~.
\end{equation}
With introduction of two variables, $\xi = x-ct $ and $\nu= x+ct$, now we can write:
\begin{equation}
	0 = \partial_{\xi}\partial_{\nu} \phi ~.
\end{equation}
Therefor the wave equation has the general solution
\begin{equation}
	\phi(t,x) =f(\xi) + g(\nu)=  f(x-ct) + g(x+ct)
\end{equation}
which is a composition of a forward and a backward propagating elementary wave.
As boundary condition we assume the initial time value and its derivative:
\begin{align}
	\phi_0(x) = \phi(0,x) &= f(x) + g(x) =: A(x) \\
	\big(\partial_t \phi(0,x)\big) &= cf'(x) - cg'(x) =: B(x)~.
	\label{eq:iv-time}
\end{align}
Integrating eq. \ref{eq:iv-time} results in
\begin{equation}
	f(x)-g(x)= \int_{x_0}^x B(s)\,\mathrm{d}s
\end{equation}
so that the separated solution is given by (here $x_0 = x(t=0)$ )
\begin{align}
	f(x) & =\frac{1}{2}
	\left( A(x)+\frac{1}{c}\int_{x_0}^x B(s)\,\mathrm{d}s \right)\\
	g(x) & = \frac{1}{2}
	\left( A(x)-\frac{1}{c} \int_{x_0}^x B(s)\,\mathrm{d}s \right) ~.
\end{align}
Putting these together yields the general solution in dependence of the initial values, the d'Alembert's formula:
\begin{equation}
	\phi(t,x) =\frac{1}{2}
	\left( A(x-ct) +A(x+ct) + \frac{1}{c}\int_{x-ct}^{x+ct} B(s)\,
	\mathrm{d}s \right)
\end{equation}
This means, for given initial parameters the wave's evolution in time is clearly defined.


\subsection{Rewriting the WE as two differential equations out of one}
Now we introduce the time derivative of the field as a new variable: $\Pi (t,x) = \partial_t \phi$. Thereby the initial wave equation eq. \ref{eq:we} can be rewritten as a wave equation including the first-order derivative in time and second-order derivative in space:
\begin{equation}
	\partial_t \Pi = \partial_{tt} \phi =  c^2 \partial_{xx}\phi ~.
\end{equation}
Now one can, in accordance to the notation we introduced above, regard the field's derivative at the initial time $\Pi(0,x)$ as $B(x)$.
Typical values for this initial velocity $\Pi (0,x)$ are either 0 or a constant.

\paragraph{Example 1: $\Pi(0,x) = 0$}
If we consider the field's initial velocity to be zero, $\Pi(0,x) = 0$ almost everywhere, this means the wave roughly is a pulse. Furthermore, if we assume no spatial boundaries the pulse will dominate the shape of the wave as time moves forward.
\paragraph{Example 2: $\Pi(0,x) = \text{const.}$}
If we consider the initial velocity to be a constant, the wave function is at rest in a suitable inertial frame of reference. Hence, there will be no wave propagation.



\subsection{Rewriting the WE as a fully first order system / eigenvalue problem}

Similar to the definition of the velocity variable $\Pi(t,x)$ we now introduce the slope variable as the spatial derivative of the field: $\chi = \partial_x \phi$.

With the \emph{state vector} $\vec{u} = \{ \phi, \Pi , \chi \}$ and the inhomogeneity source term $S$ we can write the wave equation as an eigenvalue problem in matrix form:
\begin{equation}
	\partial_t \vec{u} + \hat{M} \partial_x \vec{u} = \vec{S}
\end{equation}
Solutions to this problem are called \emph{eigenmodes}.


\paragraph{Example 1:}